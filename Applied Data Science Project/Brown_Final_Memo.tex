\documentclass[12pt]{article}
\usepackage{lingmacros}
\title{Final Project Memo}
\author{Matthew Brown}
\date{\today}
\begin{document}
\maketitle
\author
\section*{Abstract}
There has been much discussion lately on the influence money has here in Washington. While many have studied the degree to which political contributions affect legislative behavior, there has been remarkably little research on the actual material effects of money in campaigns. While fundraising is generally accepted to be critically important, there has been little work done to quantify its influence. 
\section*{Goals}
\begin{itemize}
  \item to quantify the effect that money has on elections
  \item to identify any historical trends in the effectiveness of money
  \item to create a tool for candidates to use which will enable them to determine how much money they will need to raise based on their opponent's fundraising, partisan makeup of the district, previous election results, and the incumbency of the opponent
\end{itemize}

\section*{Procedure}
Data will be collected for congressional races from the earliest data such data can be obtained to the 2018 midterms. The data will be manipulated and cleaned to include appropriate data points for all general election candidates: incumbent status, political party, previous cycle election results in the district or state (if applicable), partisan voting index of the district or state (if avaliable), amount spent, amount spent by opponent, final share of the vote, and a dummy variable to identify whether or not they won. Using this data, a formula will be calculated to determine the relationship between pvi (if avaliable), the final election result, incumbency, and fundraising discrepancies between the candidates. A second formula will be calculated to determine the relationship between the previous election result in the district or state, the final election result, incumbency, and fundraising discrepancies between the candidates. A GUI will be created to allow prospective candidates to calculate how much they need to raise based on these data points.
\end{document}